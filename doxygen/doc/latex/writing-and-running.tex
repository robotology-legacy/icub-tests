\hypertarget{writing-and-running_writing-new-tests}{}\section{Writing a new test}\label{writing-and-running_writing-new-tests}
Create a folder with the name of your test case in the {\ttfamily icub-\/tests/src/} folder to keep your test codes\+:


\begin{DoxyCode}
$ mkdir icub-tests/src/example-test
\end{DoxyCode}


Create a child test class inherited from the {\ttfamily Yarp\+Test\+Case}\+:


\begin{DoxyCodeInclude}
\textcolor{comment}{/*}
\textcolor{comment}{ * iCub Robot Unit Tests (Robot Testing Framework)}
\textcolor{comment}{ *}
\textcolor{comment}{ * Copyright (C) 2015-2019 Istituto Italiano di Tecnologia (IIT)}
\textcolor{comment}{ *}
\textcolor{comment}{ * This library is free software; you can redistribute it and/or}
\textcolor{comment}{ * modify it under the terms of the GNU Lesser General Public}
\textcolor{comment}{ * License as published by the Free Software Foundation; either}
\textcolor{comment}{ * version 2.1 of the License, or (at your option) any later version.}
\textcolor{comment}{ *}
\textcolor{comment}{ * This library is distributed in the hope that it will be useful,}
\textcolor{comment}{ * but WITHOUT ANY WARRANTY; without even the implied warranty of}
\textcolor{comment}{ * MERCHANTABILITY or FITNESS FOR A PARTICULAR PURPOSE.  See the GNU}
\textcolor{comment}{ * Lesser General Public License for more details.}
\textcolor{comment}{ *}
\textcolor{comment}{ * You should have received a copy of the GNU Lesser General Public}
\textcolor{comment}{ * License along with this library; if not, write to the Free Software}
\textcolor{comment}{ * Foundation, Inc., 51 Franklin Street, Fifth Floor, Boston, MA  02110-1301  USA}
\textcolor{comment}{ */}

\textcolor{preprocessor}{#ifndef \_EXAMPLE\_TEST\_H\_}
\textcolor{preprocessor}{#define \_EXAMPLE\_TEST\_H\_}

\textcolor{preprocessor}{#include <yarp/robottestingframework/TestCase.h>}

\textcolor{keyword}{class }ExampleTest : \textcolor{keyword}{public} yarp::robottestingframework::TestCase \{
\textcolor{keyword}{public}:
    ExampleTest();
    \textcolor{keyword}{virtual} ~ExampleTest();

    \textcolor{keyword}{virtual} \textcolor{keywordtype}{bool} setup(yarp::os::Property& property);

    \textcolor{keyword}{virtual} \textcolor{keywordtype}{void} tearDown();

    \textcolor{keyword}{virtual} \textcolor{keywordtype}{void} run();
\};

\textcolor{preprocessor}{#endif //\_EXAMPLE\_TEST\_H}
\end{DoxyCodeInclude}


Implement the test case\+:


\begin{DoxyCodeInclude}
\textcolor{comment}{/*}
\textcolor{comment}{ * iCub Robot Unit Tests (Robot Testing Framework)}
\textcolor{comment}{ *}
\textcolor{comment}{ * Copyright (C) 2015-2019 Istituto Italiano di Tecnologia (IIT)}
\textcolor{comment}{ *}
\textcolor{comment}{ * This library is free software; you can redistribute it and/or}
\textcolor{comment}{ * modify it under the terms of the GNU Lesser General Public}
\textcolor{comment}{ * License as published by the Free Software Foundation; either}
\textcolor{comment}{ * version 2.1 of the License, or (at your option) any later version.}
\textcolor{comment}{ *}
\textcolor{comment}{ * This library is distributed in the hope that it will be useful,}
\textcolor{comment}{ * but WITHOUT ANY WARRANTY; without even the implied warranty of}
\textcolor{comment}{ * MERCHANTABILITY or FITNESS FOR A PARTICULAR PURPOSE.  See the GNU}
\textcolor{comment}{ * Lesser General Public License for more details.}
\textcolor{comment}{ *}
\textcolor{comment}{ * You should have received a copy of the GNU Lesser General Public}
\textcolor{comment}{ * License along with this library; if not, write to the Free Software}
\textcolor{comment}{ * Foundation, Inc., 51 Franklin Street, Fifth Floor, Boston, MA  02110-1301  USA}
\textcolor{comment}{ */}

\textcolor{preprocessor}{#include "ExampleTest.h"}
\textcolor{preprocessor}{#include <robottestingframework/dll/Plugin.h>}
\textcolor{preprocessor}{#include <robottestingframework/TestAssert.h>}

\textcolor{keyword}{using namespace }std;
\textcolor{keyword}{using namespace }robottestingframework;
\textcolor{keyword}{using namespace }yarp::os;

\textcolor{comment}{// prepare the plugin}
ROBOTTESTINGFRAMEWORK\_PREPARE\_PLUGIN(ExampleTest)

ExampleTest::ExampleTest() : yarp::robottestingframework::TestCase(\textcolor{stringliteral}{"ExampleTest"}) \{
\}

ExampleTest::~ExampleTest() \{ \}

\textcolor{keywordtype}{bool} ExampleTest::setup(yarp::os::Property &property) \{

    \textcolor{comment}{// initialization goes here ...}
    \textcolor{comment}{//updating the test name}
    \textcolor{keywordflow}{if}(property.check(\textcolor{stringliteral}{"name"}))
        setName(property.find(\textcolor{stringliteral}{"name"}).asString());

    \textcolor{keywordtype}{string} example = \textcolor{keyword}{property}.check(\textcolor{stringliteral}{"example"}, Value(\textcolor{stringliteral}{"default value"})).asString();

    ROBOTTESTINGFRAMEWORK\_TEST\_REPORT(Asserter::format(\textcolor{stringliteral}{"Use '%s' for the example param!"},
                                       example.c\_str()));
    \textcolor{keywordflow}{return} \textcolor{keyword}{true};
\}

\textcolor{keywordtype}{void} ExampleTest::tearDown() \{
    \textcolor{comment}{// finalization goes her ...}
\}

\textcolor{keywordtype}{void} ExampleTest::run() \{

    \textcolor{keywordtype}{int} a = 5; \textcolor{keywordtype}{int} b = 3;
    ROBOTTESTINGFRAMEWORK\_TEST\_CHECK(a<b, \textcolor{stringliteral}{"a smaller then b"});
    ROBOTTESTINGFRAMEWORK\_TEST\_CHECK(a>b, \textcolor{stringliteral}{"a bigger then b"});
    ROBOTTESTINGFRAMEWORK\_TEST\_CHECK(a==b, \textcolor{stringliteral}{"a equal to b"});

    \textcolor{comment}{// add more}
    \textcolor{comment}{// ...}
\}

\end{DoxyCodeInclude}


Notice\+: The {\ttfamily R\+O\+B\+O\+T\+T\+E\+S\+T\+I\+N\+G\+F\+R\+A\+M\+E\+W\+O\+R\+K\+\_\+\+T\+E\+S\+T\+\_\+\+C\+H\+E\+CK}, {\ttfamily R\+O\+B\+O\+T\+T\+E\+S\+T\+I\+N\+G\+F\+R\+A\+M\+E\+W\+O\+R\+K\+\_\+\+T\+E\+S\+T\+\_\+\+R\+E\+P\+O\+RT} do N\+OT threw any exception and are used to add failure or report messages to the result collector. Instead, all the macros which include {\ttfamily \+\_\+\+A\+S\+S\+E\+R\+T\+\_\+} within their names (e.\+g., {\ttfamily R\+O\+B\+O\+T\+T\+E\+S\+T\+I\+N\+G\+F\+R\+A\+M\+E\+W\+O\+R\+K\+\_\+\+A\+S\+S\+E\+R\+T\+\_\+\+F\+A\+IL}) throw exceptions which prevent only the current test case (Not the whole test suite) of being proceed. The error/failure messages thrown by the exceptions are caught. (See \href{http://robotology.github.io/robot-testing-framework/documentation/TestAssert_8h.html}{\tt {\itshape Basic Assertion macros}}).

The report/assertion macros store the source line number where the check/report or assertion happen. To see them, you can run the test case or suite with {\ttfamily -\/-\/detail} parameter using the {\ttfamily robottestingframework-\/testrunner} (See \href{http://robotology.github.io/robot-testing-framework/documentation/robottestingframework-testrunner.html}{\tt {\itshape Running test case plug-\/ins using robottestingframework-\/testrunner}}).

Create a cmake file to build the plug-\/in\+:


\begin{DoxyCodeInclude}
\textcolor{preprocessor}{# iCub Robot Unit Tests (Robot Testing Framework)}
\textcolor{preprocessor}{#}
\textcolor{preprocessor}{# Copyright (C) 2015-2019 Istituto Italiano di Tecnologia (IIT)}
\textcolor{preprocessor}{#}
\textcolor{preprocessor}{# This library is free software; you can redistribute it and/or}
\textcolor{preprocessor}{# modify it under the terms of the GNU Lesser General Public}
\textcolor{preprocessor}{# License as published by the Free Software Foundation; either}
\textcolor{preprocessor}{# version 2.1 of the License, or (at your option) any later version.}
\textcolor{preprocessor}{#}
\textcolor{preprocessor}{# This library is distributed in the hope that it will be useful,}
\textcolor{preprocessor}{# but WITHOUT ANY WARRANTY; without even the implied warranty of}
\textcolor{preprocessor}{# MERCHANTABILITY or FITNESS FOR A PARTICULAR PURPOSE.  See the GNU}
\textcolor{preprocessor}{# Lesser General Public License for more details.}
\textcolor{preprocessor}{#}
\textcolor{preprocessor}{# You should have received a copy of the GNU Lesser General Public}
\textcolor{preprocessor}{# License along with this library; if not, write to the Free Software}
\textcolor{preprocessor}{# Foundation, Inc., 51 Franklin Street, Fifth Floor, Boston, MA  02110-1301  USA}


\textcolor{keywordflow}{if}(NOT DEFINED CMAKE\_MINIMUM\_REQUIRED\_VERSION)
  cmake\_minimum\_required(VERSION 3.5)
endif()

project(ExampleTest)

\textcolor{preprocessor}{# add the required cmake packages}
find\_package(RobotTestingFramework 2 COMPONENTS DLL)
find\_package(YARP 3.3.0 COMPONENTS os robottestingframework)

\textcolor{preprocessor}{# add the source codes to build the plugin library}
add\_library($\{PROJECT\_NAME\} MODULE ExampleTest.h
                                   ExampleTest.cpp)

# add required libraries
target\_link\_libraries($\{PROJECT\_NAME\} RobotTestingFramework::RTF
                                      RobotTestingFramework::RTF\_dll
                                      YARP::YARP\_os
                                      YARP::YARP\_init
                                      YARP::YARP\_robottestingframework)

# \textcolor{keyword}{set} the installation options
install(TARGETS $\{PROJECT\_NAME\}
        EXPORT $\{PROJECT\_NAME\}
        COMPONENT runtime
        LIBRARY DESTINATION lib)

\end{DoxyCodeInclude}


Call your cmake file from the {\ttfamily icub-\/test/\+C\+Make\+Lists.\+txt} to build it along with the other other test plugins. To do that, adds the following line to the {\ttfamily icub-\/test/\+C\+Make\+Lists.\+txt}


\begin{DoxyCode}
\textcolor{preprocessor}{# Build example test}
\textcolor{preprocessor}{add\_subdirectory(src/example-test)}
\end{DoxyCode}


Please check the {\ttfamily icub-\/tests/example} folder for a template for developing tests for the i\+Cub.\hypertarget{writing-and-running_running_single_test_case}{}\section{Running a single test case}\label{writing-and-running_running_single_test_case}
As it is documented here (\href{http://robotology.github.io/robot-testing-framework/documentation/robottestingframework-testrunner.html}{\tt {\itshape Running test case plug-\/ins using robottestingframework-\/testrunner}}) you can run a single test case or run it with the other tests using a test suite. For example, to run a single test case\+:


\begin{DoxyCode}
robottestingframework-testrunner --verbose --test plugins/ExampleTest.so  --param \textcolor{stringliteral}{"--name MyExampleTest"}
\end{DoxyCode}


Notice that this test require the {\ttfamily yarpserver} to be running and it contains tests that are programmed to succeed and some that are programmed to fail.

or to run the i\+Cub\+Sim camera test whith the test configuration file\+:


\begin{DoxyCode}
robottestingframework-testrunner --verbose --test plugins/CameraTest.so --param \textcolor{stringliteral}{"--from camera\_right.ini"} -
      -environment \textcolor{stringliteral}{"--robotname icubSim"}
\end{DoxyCode}


This runs the icub\+Sim right-\/camera test with the parameters specified in the {\ttfamily right\+\_\+camera.\+ini} which can be found in {\ttfamily icub-\/tests/suites/contexts/icub\+Sim} folder. This test assumes you are running {\ttfamily yarpserver} and the i\+Cub simulator (i.\+e. {\ttfamily i\+Cub\+\_\+\+S\+IM}).

Notice that the environment parameter {\ttfamily -\/-\/robotname icub\+Sim} is used to locate the correct context (for this examples is {\ttfamily icub\+Sim}) and also to update the variables loaded from the {\ttfamily right\+\_\+camera.\+ini} file.\hypertarget{writing-and-running_running_multiple_tests}{}\section{Running multiple tests using a test suite}\label{writing-and-running_running_multiple_tests}
You can update one of the existing suite X\+ML files to add your test case plug-\/in and its parameters or create a new test suite which keeps all the relevant test cases. For example the {\ttfamily basic-\/icub\+Sim.\+xml} test suite keeps the basic tests for cameras and motors\+:


\begin{DoxyCode}
<?xml version=\textcolor{stringliteral}{"1.0"} encoding=\textcolor{stringliteral}{"UTF-8"}?>

<suite name=\textcolor{stringliteral}{"Basic Tests Suite"}>
    <description>Testing robot\textcolor{stringliteral}{'s basic features</description>}
\textcolor{stringliteral}{    <environment>--robotname icubSim</environment>}
\textcolor{stringliteral}{    <fixture param="--fixture icubsim-fixture.xml"> yarpmanager </fixture>}
\textcolor{stringliteral}{}
\textcolor{stringliteral}{    }
\textcolor{stringliteral}{    <test type="dll" param="--from right\_camera.ini"> CameraTest </test>}
\textcolor{stringliteral}{    <test type="dll" param="--from left\_camera.ini"> CameraTest </test>}
\textcolor{stringliteral}{}
\textcolor{stringliteral}{    }
\textcolor{stringliteral}{    <test type="dll" param="--from test\_right\_arm.ini"> MotorTest </test>}
\textcolor{stringliteral}{    <test type="dll" param="--from test\_left\_arm.ini"> MotorTest </test>}
\textcolor{stringliteral}{</suite>}
\end{DoxyCode}


Then you can run all the test cases from the test suite\+:


\begin{DoxyCode}
robottestingframework-testrunner --verbose --suite icub-tests/suites/basics-icubSim.xml
\end{DoxyCode}


The {\ttfamily robottestingframework-\/testrunner}, first, launches the i\+Cub simulator and then runs all the tests one after each other. After running all the test cases, the {\ttfamily tesrunner} stop the simulator. If the i\+Cub simulator crashes during the test run, the {\ttfamily robottestingframework-\/testrunner} re-\/launchs it and continues running the remaining tests.

How {\ttfamily robottestingframework-\/testrunner} knows that it should launch the i\+Cub simulator before running the tests? Well, this is indicated by {\ttfamily $<$fixture param=\char`\"{}-\/-\/fixture icubsim-\/fixture.\+xml\char`\"{}$>$ yarpmanager $<$/fixture$>$}. The {\ttfamily robottestingframework-\/testrunner} uses the {\ttfamily yarpmanager} fixture plug-\/in to launch the modules which are listed in the {\ttfamily icubsim-\/fixture.\+xml}. Notice that all the fixture files should be located in the {\ttfamily icub-\/tests/suites/fixtures} folder. 